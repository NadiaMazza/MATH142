\documentclass[12pt]{article}

\usepackage{amsmath,amssymb,amsthm}
\usepackage{geometry}
\usepackage{graphicx}
\usepackage{makeidx}
\usepackage{hyperref}
\usepackage{amsopn}
\usepackage[curve,matrix,arrow]{xy}


\renewcommand\familydefault{cmss}
\newcommand{\qbox}[1]{\quad\hbox{#1}\quad}

\def\dst{\displaystyle}

%thms & defs
\newtheorem{thm}{Theorem}[subsection]
\newtheorem{lemma}[thm]{Lemma}
\newtheorem{cor}[thm]{Corollary}
\newtheorem{prop}[thm]{Proposition}

\theoremstyle{definition}
\newtheorem{defn}[thm]{Definition}
\newtheorem{nota}[thm]{Notation}
\newtheorem{example}[thm]{Example}
\newtheorem{examples}[thm]{Examples}
\newtheorem{remark}[thm]{Remark}





%counters
\renewcommand\theenumi{\roman{enumi}}
\renewcommand\theenumii{\alph{enumii}}
%\renewcommand{\thesection}{\Roman{section}}
\newcounter{ex}\renewcommand\theex{\arabic{ex}}
\newenvironment{exo}{\begin{flushleft}%
\textbf{\refstepcounter{ex}Exercise~\theex.}}{\end{flushleft}}

%special text in formulae


\newcommand{\dx}[1]{\frac{\mathrm d}{\mathrm d{#1}}}
\newcommand{\ddx}[2]{\frac{\mathrm d{#1}}{\mathrm d{#2}}}
\def\dd{\operatorname{d}\!}
\def\sech{\operatorname{sech}}
\def\R{\mathbb R}
\def\C{\mathbb C}


\newcommand{\mps}{$\mathrm{ms}^{-1}$}
\newcommand{\mpss}{$\mathrm{ms}^{-2}$}
\setlength{\voffset}{-1.6cm}
\setlength{\hoffset}{-1cm} 
\setlength{\textwidth}{16.7cm}
\setlength{\textheight}{24cm}

\makeindex

\begin{document}

\begin{center}
{\Large{\bf MATH142: Engineering Mathematics II}}\\
{\large{\bf Weeks 6--10 -- Michaelmas 2019-20}}\\
Lecturer: Dr Nadia Mazza
(\textsf{n.mazza@lancaster.ac.uk}) \quad 
Office: Fylde B41
\end{center}

\tableofcontents
\vfill

\newpage
\section{Housekeeping}\label{sec:housekeeping}

\subsection{Syllabus}

\begin{itemize}\item
  The limit of a function of a real variable.
\item Continuity of a function of a real variable.
\item The definition of the derivative as a limit.
\item The graphical interpretation of the derivative.
\item Rules of differentiation.
\item Derivatives of standard functions.
\item Higher-order derivatives.
\item Stationary points and their classification.
\item Techniques of differentiation.
\item L'H\^opital's rule.
\item Definite and indefinite integrals.
\item Integration as antiderivative.
\item Integration of standard functions.
\item Techniques of integration.
\item Approximation of definite integrals using Taylor and Maclaurin series.
\item Selection of applications of integration.
\end{itemize}

\subsection{Recommended text books}
\begin{itemize}
\item A. Croft, R. Davison, M. Hargreaves and J. Flint, {\em
  Engineering Mathematics: A Foundation for Electronic, Electrical,
  Communications and Systems Engineers}, fourth edition, 2013, Pearson
  Education Ltd, ISBN 978­0­273­71977­9.
\item
K.A. Stroud with D. J. Booth, {\em Engineering Mathematics}, seventh
edition, 2013, Palgrave Macmillan Ltd, ISBN 978­1­137­03120­4
\end{itemize}

%%%%%%%%%%%%%%%%%%%
\subsection{Lectures and class exercises}


The notes for this course are complete and, together with the workshop
questions and assessed exercises, provide all the information you
need. It is recommended that you take your own notes from the lectures
and additional material. 
Examples are given in lectures and in the notes, and further practice
will be gained by working through exercises during workshops,
self-study and lecture time. 
You must actively participate in the
activities as directed by your lecturer and tutors.
There are 3 x 1h lectures per week and two workshops per week, as
indicated in your timetable. Further details are on the Moodle page
for the module.

%%%%%%%%%%%%%%%%%%%%%%%%%%%%%

\subsection{Learning outcomes}

{\bf Subject specific}

On successful completion of this module, the student should be able
to: 

\begin{enumerate}
\item understand the meaning of a derivative, both algebraically and
graphically; 
\item differentiate simple functions, products and quotients
and use a variety of techniques; 
\item use L'H\^opital's rule to find limits of functions;
\item understand the meaning of definite and indefinite integrals;
\item integrate expressions directly, by parts and by substitution; 
\item apply integration to calculate physical quantities; 
\item find Taylor and Maclaurin series, and use these to estimate 
definite integrals;
\end{enumerate}

\vspace{.3cm}
{\bf General knowledge, understanding and skills}

On successful completion of this module, the student should be able to
use a range of mathematical techniques from differential and integral
calculus to solve engineering problems, 
present their work in a clear and coherent manner
Demonstrate basic problem solving skills 

\subsection{Assessment}

\begin{tabular}{l}
20\% equally weighted weekly coursework \\
20\% end-of-module test\\
60\% end-of-year examination
\end{tabular}

\smallskip
The weekly coursework and the end-of-module test form the 40\%
coursework part of your final mark for the module.


\smallskip
{\bf MATH142 coursework must be submitted no later than 12 NOON WEDNESDAY.} 

Further details on the MATH142 Moodle page.

\smallskip
Any late
submission is subject to penalties. Under no circumstances can
coursework be accepted after the answers are given out at the
following workshop or published online. 
Feedback shall be returned promptly.

\smallskip
{\bf End-of-module test}

\smallskip
The end-of-module test will take place in week 10, and you will be informed
nearer the time about the exact date, time and venue. Please refer to
your timetable and to the MATH142 Moodle page.
It is {\bf your responsibility} to complete and submit the test on
time. 
If you are ill, a medical certificate must be
provided to the Teaching Office.
{\bf If written confirmation of the reason for your absence is not
received, you will be given a mark of zero.}

\smallskip
{\bf Examination}

\smallskip
The examination will take place during the main exam session in
summer. You will be informed by Student Registry of the examination
processes. It is {\bf your responsibility} to attend these
examinations on time unless prevented by ill health.


\end{document}
